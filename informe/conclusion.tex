\section{Conclusión}
Hemos presentado los componentes básicos de un Mark-Sweep GC. Analizamos dos alternativas para la implementación del marcado, tanto desde el punto de vista teórico como desde el práctico. Concluimos que no hay una implementación superior a la otra para todo escenario posible, sino que depende de los requerimientos del contexto. La característica de las dos implementaciones es que el talón de Aquiles de una es el punto fuerte de la otra, lo cual lleva a pensar que podría ser interesante la búsqueda de una tercera opción que sea, en algún sentido, intermedia, haciendo un balance entre complejidad espacial y temporal.